{\IRSTLM} allows the use of class and chunk LMs, and a special
handling of input tokens which are concatenation of $N \ge 1$ fields separated
by the character \#, e.g.

\begin{verbatim}
   word#lemma#part-of-speech#word-class
\end{verbatim}

\noindent The processing is guided by the format of the file passed to
Moses or {\tt compile-lm}: if it contains just the LM, either in textual or
binary format, it is treated as usual; otherwise, it is supposed to have
the following format:

\begin{verbatim}
LMMACRO <lmmacroSize> <selectedField> <collapse>
<lmfilename>
<mapfilename>
\end{verbatim}

\noindent where:
\begin{verbatim}
 LMMACRO is a reserved keyword
 <lmmacroSize> is a positive integer
 <selectedField> is an integer >=-1
 <collapse> is a boolean value (true, false)
 <lmfilename> is a file containing a LM (format compatible with {\IRSTLM})
 <mapfilename> is an (optional) file with a (one|many)-to-one map
\end{verbatim}

\noindent The various cases are discussed with examples in the
following. Data used in those examples can be found in the directory {\tt
example/chunkLM/} which represents the relative path for all the parameters
of the referred commands.  Note that texts with different tokens (words,
POS, word\#POS pairs...) used either as input or for training LMs are all
derived from the same multifield texts in order to allow direct comparison
of results.

\subsection{Field selection}

The simplest case is that of the LM in {\tt <lmfilename>} referring just to
one specific field of the input tokens. In this case, it is possible to
specify the field to be selected before querying the LM through the integer
{\tt <selectedField>} ($0$ for the first filed, $1$ for the
second...). With the value $-1$, no selection is applied and the LM is
queried with n-grams of whole strings.  The other parameters are set as:

\begin{verbatim}
 <lmmacroSize> : set to the size of the LM in <lmfilename>
 <collapse>    : false
\end{verbatim}

\noindent The third line optionally reserved to {\tt <mapfilename>} does not exist.

\bigskip
\noindent
Examples:

\bigskip
\noindent
\thesubsection.a) selection of the second field:
\begin{verbatim}
$> compile-lm --eval test/test.w-micro cfgfile/cfg.2ndfield
%% Nw=126 PP=2.68 PPwp=0.00 Nbo=0 Noov=0 OOV=0.00%
\end{verbatim}

\noindent
\thesubsection.b) selection of the first field:
\begin{verbatim}
$> compile-lm --eval test/test.w-micro cfgfile/cfg.1stfield
%% Nw=126 PP=9.71 PPwp=0.00 Nbo=76 Noov=0 OOV=0.00%
\end{verbatim}

\noindent The result of the latter case is identical to that obtained with
the standard configuration involving just words:

\bigskip
\noindent
\thesubsection.c) usual case on words:
\begin{verbatim}
$> compile-lm --eval test/test.w lm/train.en.blm 
%% Nw=126 PP=9.71 PPwp=0.00 Nbo=76 Noov=0 OOV=0.00%
\end{verbatim}


\subsection{Class LMs}

Possibly, a many-to-one or one-to-one map can be passed through the
{\tt <mapfilename>} parameter which has the simple format:

\begin{verbatim}
w1 class(w1)
w2 class(w2)
 ...
wM class(wM)
\end{verbatim}


\noindent The map is applied to each component of ngrams before the LM
query. Examples:
\bigskip

\noindent \thesubsection.a) map applied to the second field:
\begin{verbatim}
$> compile-lm --eval test/test.w-micro cfgfile/cfg.2ndfld-map
%% Nw=126 PP=16.40 PPwp=0.00 Nbo=33 Noov=0 OOV=0.00%
\end{verbatim}



\noindent \thesubsection.b) just to assess the correctness of the (16.2.a) result:
\begin{verbatim}
$> compile-lm --eval test/test.macro lm/train.macro.blm
%% Nw=126 PP=16.40 PPwp=0.00 Nbo=33 Noov=0 OOV=0.00%


\end{verbatim}


\subsection{Chunk LMs}

A particular processing is performed whenever fields are supposed to
correspond to microtags, i.e. the per-word projections of chunk labels. By
means of the {\tt <collapse>} parameter, it is possible to activate a
processing aiming at collapsing the sequence of microtags defining a
chunk. The chunk LM is then queried with ngrams of chunk labels, in an
asynchronous manner with respect to the sequence of words, as in general
chunks consist of more words.

\noindent
The collapsing operation is automatically activated if the sequence of
microtags is:

\begin{verbatim}
 TAG( TAG+ TAG+ ... TAG+ TAG)
\end{verbatim}

\noindent
Such a sequence is collapsed into a single chunk label (let us say {\tt
CHNK}) as long as {\tt TAG(}, {\tt TAG+} and {\tt TAG)} are all mapped into
the same label {\tt CHNK}. The map into different labels or a different
use/position of characters $($, $+$ and $)$ in the lexicon of tags prevent
the collapsing operation even if {\tt <collapse>} is set to {\tt true}. Of
course, if {\tt <collapse>} is {\tt false}, no collapse is attempted.

\paragraph{Warning:} In this context, it assumes an important role the parameter {\tt
<lmmacroSize>}: it defines the size of the n-gram before the collapsing
operation, that is the number of microtags of the actually processed
sequence. {\tt <lmmacroSize>} should be large enough to ensure that after
the collapsing operation, the resulting n-gram of chunks is at least of the
size of the LM to be queried (the {\tt <lmfilename>}). As an example,
assuming {\tt <lmmacroSize>=6}, {\tt <selectedField>=1}, {\tt
<collapse>=true} and 3 the size of the chunk LM, the following input

\begin{verbatim}
 on#PP average#NP( 30#NP+ -#NP+ 40#NP+ cm#NP)
\end{verbatim}

\noindent will yield to query the LM with just the bigram {\tt (PP,NP)},
instead of a more informative trigram; for this particular case, the value
6 for {\tt <lmmacroSize>} is not enough.  On the other side, for efficiency
reasons, it cannot be set to an unlimited valued. A reasonable value could
derive from the average number of microtags per chunk (2-3), which means
setting {\tt <lmmacroSize>} to two-three times the size of the LM in {\tt
<lmfilename>}.  Examples:
\bigskip

\noindent \thesubsection.a) second field, micro$\rightarrow$macro map, collapse:
\begin{verbatim}
$> compile-lm --eval test/test.w-micro cfgfile/cfg.2ndfld-map-cllps
%% Nw=126 PP=1.84 PPwp=0.00 Nbo=0 Noov=0 OOV=0.00%

$> compile-lm --eval test/test.w-micro cfgfile/cfg.2ndfld-map-cllps -d=1
%% Nw=126 PP=1.83774013 ... OOV=0.00% logPr=-33.29979642

\end{verbatim}

\noindent
\thesubsection.b) whole token,  micro$\rightarrow$macro map, collapse:
\begin{verbatim}
$> compile-lm --eval test/test.micro cfgfile/cfg.token-map-cllps
%% Nw=126 PP=1.84 PPwp=0.00 Nbo=0 Noov=0 OOV=0.00%
\end{verbatim}

\noindent
\thesubsection.c)  whole token,  micro$\rightarrow$macro map, NO collapse:
\begin{verbatim}
$> compile-lm --eval test/test.micro cfgfile/cfg.token-map
%% Nw=126 PP=16.40 PPwp=0.00 Nbo=0 Noov=0 OOV=0.00%
\end{verbatim}
\noindent Note that the configuration (16.3.c) gives the same result of that in
example (16.2.b), as they are equivalent.

\bigskip
\noindent
\thesubsection.d) As an actual example related to the ``warning'' note
reported above, the following configuration with usual LM:

\begin{verbatim}
$> compile-lm --eval test/test.chunk lm/train.macro.blm -d=1
Nw=73 PP=2.85754443 ... OOV=0.00000000% logPr=-33.28748842
\end{verbatim}

\noindent not necessarily yields the same log-likelihood ({\tt logPr}) nor the same perplexity ({\tt PP}) of case (16.3.a).
In fact, concerning {\tt PP}, the length of the input sequence is definitely different (126 tokens before collapsing, 73 after that).
Even the {\tt logPr} is different (-33.29979642 vs. -33.28748842) because in (16.3.a) some 6-grams ({\tt
<lmmacroSize>} is set to 6) after collapsing reduce to $n$-grams of size less
than 3 (the size of lm/train.macro.blm). By setting {\tt <lmmacroSize>} to
a larger value (e.g. 8), the same {\tt logPr} will be computed.


